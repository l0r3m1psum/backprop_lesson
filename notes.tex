\documentclass{article}

\title{Back-propagation Lecture Notes}
\author{Diego Bellani}
\date{30 September 2024}

\usepackage{graphics}

% https://tex.stackexchange.com/a/284054
\newcommand{\argmin}{\mathop{\mathrm{arg\,min}}}
\newcommand{\trace}{\mathrm{tr}}
\newcommand{\expected}{\mathop\mathsf E}
\newcommand{\partialfrac}[2]{\frac{\partial #1}{\partial #2}}
\newcommand{\Diag}{\mathrm{Diag}} % diagonalize vector to matrix
\newcommand{\diag}{\mathbf{diag}} % diagonalize matrix to vector
\newcommand{\vect}{\mathbf{vec}} % vectorize
\newcommand{\krone}{\otimes} % Kronecker product
\newcommand{\hadam}{\odot} % Hadamard product
% https://tex.stackexchange.com/a/163041
\newcommand{\dif}{\mathop{}\!\mathsf d} % differential
\newcommand{\der}{\mathop{}\!\mathsf D} % derivative
\newcommand{\pluseq}{\mathrel{{+}{=}}}
\renewcommand\vec[1]{\mathbf{#1}}
% https://tex.stackexchange.com/a/638348
% http://latexref.xyz/_005cDeclareGraphicsRule.html
\DeclareGraphicsRule{*}{mps}{*}{}

\begin{document}
\maketitle

\section{Inroduction}

The task performed by a neural network is determined by the values of its weights.
To find the weight needed to perform a certain task we use a series of examples
(the training data,) together with a loss function that determines how far we are
from performing said task on the given examples. In order to get our network
closer to perform the task we want on the examples we use a \emph{gradient}
descent algorithm.

We need a way to calculate the gradient of the loss of the neural network on
the example data w.r.t. the weights of the network.

In the context of neural networks it is often the case that we modify the model
itself to make the task learnable with the available data. This is takes a lot of
trial and error, therefore we need an automated way to calculate said gradient.

A possible way to do so is to is \emph{numerical differentiation}. That is, in
its most basic form, to approximate the gradient using the definition of the
derivative, i.e. \[\lim_{h \to 0} \frac{f(x+h)-f(x)} h.\]

Let us try to see if this is a viable option with an example, let
\begin{equation}
f(x_1, x_2) = \log(x_1)+x_1x_2-\sin(x_2) \label{eqn:example}
\end{equation}
and say that we want to calculate \(\nabla{f(2,5)}.\) We now want to confront
the value that we would obtain from the numerical approximation against the
formula that we would find by hand. Since \(\nabla{f(2,5)}\) has two elements,
let us start from the second. We know from basic calculus that
\(\partialfrac{f(x_1,x_2)}{x_2} = x_1-\cos(x2).\) Now to use the limit
definition in our numerical approximation, since computers cannot take limits,
we are going to use a small number instead for \(h.\) Let us write a quick
implementation below.

\begin{verbatim}
import numpy
def f(x1,x2): return numpy.log(x1) + x1*x2 - numpy.sin(x2)
x1, x2, h = numpy.array([2.,5.,1e-5], dtype=numpy.float32)
d_f_x2 = (f(x1,x2+h)-f(x1,x2))/h
print(numpy.absolute(x1-numpy.cos(x2) - d_f_x2))
\end{verbatim}

The code above prints \texttt{0.09509146}, this means
that the numerical approximation, in this simple example, is already off by one
decimal point. This is not acceptable, not to mention that to find a gradient
of \(n\) elements it requires \(O(n)\) evaluations of the function which may be
very expensive.

Another way to automate the calculation of the gradient is to use
\emph{symbolic differentiation}. That is to treat the expression of the loss
as an data structure\footnote{A directed acyclic graph.} and write an algorithm
that applies the rules of calculus for us. Some people may tell you that this
approach does not scale due to \emph{expression swell}, that is to say that the
na\"ive application of calculus rules can lead to an explosion in the size of
the expression and in a lot of redundant calculations. E.g
\(\partialfrac{\sin(e^x)} x = e^x \cos(e^x),\) as you can see in a na\"ive
implementation we have to evaluate and store the node representing \(e^x\) two
times. This problem can be solved with \emph{common sub-expression
elimination}. The real problem lies in the fact that we have to use another
language to specify our neural network and in how to differentiate through
control-flow\footnote{Again this is not strictly true. We could use our own
programming language as the language to specify the expression that we want to
differentiate and than use \emph{reflection} to transform it in a suitable data
structure. With this approach we can also make control-flow structures work but
if you get this far you have to make some philosophy about what you mean by
symbolic differentiation\dots~\cite{equiv}.}.

In the rest of this document we are going to introduce the
\emph{back-propagation} algorithm which is a special case of \emph{reverse-mode
automatic differentiation}, which is one of the differentiation modes studied
by the field of \emph{automatic differentiation}~\cite{survey,griewank}.

\section{Scalar Back-propagation}

To recap, we need a an efficient way to calculate gradient, at \emph{machine
precision}, of a function written in a programming language, supporting all of
its control-flow constructs. This task may seem daunting but we will show how
it can be done, starting from the general idea, then describing how it is
usually implemented and how this gives us for free support for deriving through
control-flow.

Let us start by considering equation~\ref{eqn:example}, in particular how we can
use the chain rule to calculate the two elements of its gradient. But first a
bit of notation, we are going to decompose equation~\ref{eqn:example} in its
elementary operations as shown of the left side of figure~\ref{fig:example}.
We can then express the two elements of the gradient as
\begin{eqnarray*}
\partialfrac{f(x_1,x_2)}{x_1}
&=& \partialfrac{v_5}{v_4}\left(\partialfrac{v_4}{v_1}\partialfrac{v_1}{v_{-1}}
	+ \partialfrac{v_4}{v_2}\partialfrac{v_2}{v_{-1}}\right) \\
&=& \partialfrac{v_5}{v_4}\partialfrac{v_4}{v_1}\partialfrac{v_1}{v_{-1}}
	+ \partialfrac{v_5}{v_4}\partialfrac{v_4}{v_2}\partialfrac{v_2}{v_{-1}} \\
\partialfrac{f(x_1,x_2)}{x_2}
&=& \partialfrac{v_5}{v_3}\partialfrac{v_3}{v_0}
	+ \partialfrac{v_5}{v_4}\partialfrac{v_4}{v_2}\partialfrac{v_2}{v_0}
\end{eqnarray*}
and in figure~\ref{fig:dag} we have the dependencies among the variables used
to evaluate the expression.

\begin{figure}
\begin{minipage}{.5\textwidth}
\begin{eqnarray*}
v_{-1} &=& x_1 \\
v_0 &=& x_2 \\
v_1 &=& \log(v_{-1}) \\
v_2 &=& v_{-1}v_0 \\
v_3 &=& \sin(v_0) \\
v_4 &=& v_1+v_2 \\
v_5 &=& v_4-v_3 \\
f(x_1,x_2) &=& v_5,
\end{eqnarray*}
\end{minipage}
\begin{minipage}{.5\textwidth}
\renewcommand{\partialfrac}[2]{\textstyle\frac{\partial#1}{\partial#2}}
\begin{eqnarray*}
\bar v_5 &=& 1 \\
\bar v_4 &=& \bar v_5 \partialfrac{v_5}{v_4} \\
\bar v_3 &=& \bar v_5 \partialfrac{v_5}{v_3} \\
\bar v_1 &=& \bar v_4 \partialfrac{v_4}{v_1} \\
\bar v_2 &=& \bar v_4 \partialfrac{v_4}{v_2} \\
\bar v_0 &=& \bar v_3 \partialfrac{v_3}{v_0} \\
\bar v_{-1} &=& \bar v_2 \partialfrac{v_2}{v_{-1}} \\
\bar v_0 &\pluseq& \bar v_2 \partialfrac{v_2}{v_0} \\
\bar v_{-1} &\pluseq& \bar v_1 \partialfrac{v_1}{v_{-1}}
\end{eqnarray*}
\end{minipage}
\label{fig:example}
\caption{Lorem ipsum.}
\end{figure}

\begin{figure}
\centering
\includegraphics{figures.1}
\label{fig:dag}
\caption{Graph representing the expression.}
\end{figure}

We can start by noting a few things. The paths in the graph from the source
node \(v_5\)
to the sinks \(v_{-1}\) and \(v_0\) correspond to the addends in the chain
rule, e.g.
\(\partialfrac{v_5}{v_3}\partialfrac{v_3}{v_0}\) correspond to the path
\(((v_5,v_3), (v_3,v_0)).\) We also have that we can evaluate a path (or addend
of a chain rule), from left to right, e.g. \[
\partialfrac{v_5}{v_4}\partialfrac{v_4}{v_1}\partialfrac{v_1}{v_{-1}}
= \partialfrac{v_5}{v_1}\partialfrac{v_1}{v_{-1}}
= \partialfrac{v_5}{v_{-1}},\]
that is we can evaluate it by left-multiplying the derivatives of \(v_5\)
w.r.t. every intermediate variable \(v_i\) along the path.

This are the foundamental ideas behing the back-propagation algorithm. We
can calculate the gradient incrementally, starting at the source node and going
backward, storing at each node \(v_i\) its derivative with respect to the
source \(\partialfrac{v_5}{v_i}.\)
If we let \(\bar v_i = \partialfrac{v_5}{v_i}\) we can see how this calculation
is performed on the right side of figure~\ref{fig:example}.

The way in which this is usually implemented in a programming language is by
keeping track of all the variables and operations used on them so that at the
end of the computation of the function we have the graph that we can use to
back-propagate the gradient through. In the case of Python this is done by
creating a class like the one below.

\begin{verbatim}
import dataclasses, enum
class Operation(enum.Enum): ...
@dataclasses.dataclass
class Scalar:
    value: float
    gradient: float
    operation: Operation
    parent1: Scalar
    parent2: Scalar
\end{verbatim}

To make the generation of the graph more ergonomic one can implement the dunder
methods to use the standard mathematical operation on \texttt{Scalar} objects.
It is important that before the back-propagation step the graph is
topologically sorted.

We will illustrate the way in which this
enables us to back-propagate through contol-flow with an example: when you train
a neural network you update its weight in a loop, so if your network is
represented by a static expression the graph that is used to back propagate has
always the same form\footnote{Of course the values stored in it are going to
change.}, if in the code that we use to specify the neural network we have an
\texttt{if} that take one of its branches based on a calculated value the graph
that is going to be generated is going to differ from one of the iteration of
the loop to another. E.g. \texttt{x = a+b if a > 0 else a+c}.
The back-propagation algorithm is therefore going to
propagate the gradient though a different graph every time but it does not care
about how it was generated, if by an \texttt{if}, a \texttt{for} or an
exception.

\section{Back-propagation for Vectors and Matrices}

We now have an algorithm that it is capable of differentiating any code, saddly
this is not enough for our needs\dots There are two main problems with this
approach. We are now going to illustrate them first for the case in which the
gradient is back-propagated through vector operations then for the more general
case of matrix operations.

The first one is a memory problem, to train neural networks we often
do computations with big matrices, multiplying two matrices of size \(n \times
n\) requires at leats \(\Omega(n^2)\) mathematical operations\footnote{The
naive matrix multiplication algorithm requires \(O(n^3)\) operations, there are
better ones but they cannot require less than \(\Omega(n^2)\), so we are going
to make a very rosey assumption here.}. If you consider that a single precision
floating point number occupies 4 bytes, and that to keep track of an operation
you are going to need at least 16 bytes\footnote{For a binary operation say we
allocate 4 bytes for pointers to the 2 parents, 4 bytes for the value of the
operation and another 4 for its gradient, ignoring the fact that we also need a
way to distinguish a among the various kinds of operations.} so there are 8
bytes of overhead per
operation so if you les \(n=1024\) we would be \(8 \cdot 1024^2\), i.e. 8MiB,
of overhead for just a single operation.

The second problem is one of performance, if we the values of our operations
all arround in memory it not possible to use optimized BLAS routines to
implement the linear algebra operations we care about, also due to how computer
works it is not really possible to implement them ourselves. Since what we care
about are mostly matrix operations it makes sense to perform operations
directly on vectors (and matrices) and not perform them at the granular scalar
level, we will need to add a few more bytes to keep track of the dimension of
the vector/matrix but it is going to be worth it.

Doing this introduces another problem, to see what is is let us start with a
quick refresh of multivariable calculus let \(\vec y = \vec f(\vec x),\) where
\(\vec f\) is a function of vector argument and value. Its derivative is a
matrix called the Jacobian with the following contents \[
\left[\begin{array}{ccc}
\partialfrac{y_1}{x_1} & \cdots & \partialfrac{y_m}{x_1} \\
\vdots & \ddots & \vdots \\
\partialfrac{y_1}{x_n} & \cdots & \partialfrac{y_m}{x_n} \\
\end{array}\right]\!.
\] So back propagating the gradient now requires multiplying the gradient
vector \(\vec g\) with the Jacobian at that node.

The problem is going to become evident with the following example, let \(\vec
f(\vec x) = \vec a \hadam \vec x,\) where \(\hadam\) is the element-wise
roduct. Its Jacobian is \[
J = \left[\begin{array}{ccc}
a_1 & \cdots & 0 \\
\vdots & \ddots & \vdots \\
0 & \cdots & a_n \\
\end{array}\right]\!,
\] that is a diagonal matrix with \(\vec a\) as its diagonal. Multiplying the
gradient with this matrix is the same as doing an element-wise multiplication,
that is \[
J \vec g = \vec a \hadam \vec g.
\] This is because \(J\) is \emph{sparse}, i.e. most of its elements are 0.
We can therefore implement matrix vector multiplication in a more space and
time efficient way by just storing \(\vec a\) instead of \(J.\) This is called
Jacobian-vector product (vJp) and can often be calculated implicitly in an
efficient way.

It is important to note that not all Jacobians are sparse, if we let \(\vec
f(\vec x) = A \vec x,\) where A is a matrix, then
\(\partialfrac{\vec f(\vec x)}{\vec x} = A',\) where \(A'\) is \(A\)
transposed.

So the Jacobians of many function is sparse and we can make some space and time
savings by being smart in the way in wich we store and operate on this sparse
Jacobians.

Neural networks are often trained in batches so the multiplications performed
with matrices are not in the form of a matrix and a vector but of a matrix and
a matrix and a matrix. Let \(F(X) = AX,\) what is \(\partialfrac{F(X)}{X}?\)

The answare is: its debated~\cite{notion}. Why is this important? Matrix multiplication is
arguably the most important operation in neural networks, therefore it is very
important to be able to back propagate through it. Where is the problem,
whatever \(\partialfrac{F(X)}{X}\) is it needs to contain al possible partial
derivatives of the input w.r.t. the output, for simplicity say that the input
and and the output have the same shape and are both square so
\(\partialfrac{F(X)}{X}\) needs to contain \(n \times n \times n \times n\)
numbers in it, now if \(n=1024\) then \(\partialfrac{F(X)}{X},\) for single
precision floatin point numbers, needs to occupy at least 4TiB, which as of
today no GPU on earth can store. Luckly it is sparse, we just need a way to
implicitly perform the vJp.

We are going to need a few pieces, first let \(\langle\vec x,\vec y\rangle\) be
the inner product of the vectors \(\vec x\) and \(\vec y.\) It can be proven
that \(\partialfrac{\langle\vec f(\vec x),\vec g(\vec x)\rangle}{\vec x}
= \partialfrac{\vec f(\vec x)}{\vec x} \vec g(\vec x)
	+ \partialfrac{\vec g(\vec x)}{\vec x} \vec f(\vec x),\) so if we keep one
of the two arguments costant deriving the inner product gives us the vJp. E.g.
\(\partialfrac{\langle\vec a \hadam \vec x,\vec g\rangle}{\vec x}
= \vec a \hadam \vec g.\)

Let \(\vect(A)\) be a vector made by stacking the columns of the matrix \(A\)
one on top of the other. In this way we can generalize our our defition of
inner product to matrices by letting \(\langle A,B\rangle =
\langle\vect(A),\vect(B)\rangle.\) Let \(\trace(A)\) be the sum of the diagonal
entries of a square matrix, we have that \(\langle A,B\rangle = \trace(A'B).\)
Finally it can be prooven that \(\partialfrac{\trace(X'A)}{X} =
A\)~\cite{cookbook,magnus}.

We are now ready to find the vJp of the matrix-matrix multiplication \(A X\),
let \(G\) be the matrix containing the gradients of the batch in its columns,
then: \[
\partialfrac{\langle AX,G\rangle}{X}
= \partialfrac{\trace((AX)'G)}{X}
= \partialfrac{\trace(X'A'G)}{X}
= A'G.\]

\bibliographystyle{plain}
\bibliography{bibliography}

\end{document}